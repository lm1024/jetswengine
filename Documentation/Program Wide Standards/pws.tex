\documentclass{article}
\usepackage[utf8]{inputenc}
\usepackage{parskip}
\usepackage[UKenglish]{babel}
\usepackage[margin=1in]{geometry}
\usepackage{fancyhdr}
\usepackage{multirow}
\usepackage[colorlinks=false]{hyperref}
\usepackage[super,square]{natbib}
\usepackage{float}
\usepackage[toc,page]{appendix}
\usepackage[table]{xcolor}


\usepackage{color}
\usepackage{listings}
\usepackage{graphicx}
\usepackage{caption}
\usepackage{wrapfig}
\usepackage{lscape}
\usepackage{rotating}
\usepackage{epstopdf}
\definecolor{mygreen}{RGB}{0,127,0}
\definecolor{mygray}{RGB}{100,100,100}
\definecolor{mymauve}{RGB}{100,32,255}
\definecolor{lgray}{RGB}{230,230,230}
\definecolor{pastelgreen}{RGB}{119, 190, 119}
\definecolor{pastelred}{RGB}{255, 105, 97}
\definecolor{pastelorange}{RGB}{255, 179, 71}
\lstset{ %
  frame=tb,
  backgroundcolor=\color{white},   % choose the background color; you must add \usepackage{color} or \usepackage{xcolor}
  basicstyle=\footnotesize\ttfamily,        % the size of the fonts that are used for the code
  breakatwhitespace=false,         % sets if automatic breaks should only happen at whitespace
  breaklines=true,                 % sets automatic line breaking
  captionpos=t,                    % sets the caption-position to bottom
  commentstyle=\color{mygreen},    % comment style
  deletekeywords={...},            % if you want to delete keywords from the given language
  escapeinside={\%*}{*)},          % if you want to add LaTeX within your code
  extendedchars=true,              % lets you use non-ASCII characters; for 8-bits encodings only, does not work with UTF-8
%  frame=single,                    % adds a frame around the code
  keepspaces=true,                 % keeps spaces in text, useful for keeping indentation of code (possibly needs columns=flexible)
  keywordstyle=\color{blue},       % keyword style
  language=,                 % the language of the code
  morekeywords={*,...},            % if you want to add more keywords to the set
  numbers=left,                    % where to put the line-numbers; possible values are (none, left, right)
  numbersep=5pt,                   % how far the line-numbers are from the code
  numberstyle=\tiny\color{mygray}, % the style that is used for the line-numbers
  rulecolor=\color{black},         % if not set, the frame-color may be changed on line-breaks within not-black text (e.g. comments (green here))
  showspaces=false,                % show spaces everywhere adding particular underscores; it overrides 'showstringspaces'
  showstringspaces=false,          % underline spaces within strings only
  showtabs=false,                  % show tabs within strings adding particular underscores
  stepnumber=1,                    % the step between two line-numbers. If it's 1, each line will be numbered
  stringstyle=\color{mymauve},     % string literal style
  tabsize=4,                       % sets default tabsize to 2 spaces
  aboveskip=3mm,
  belowskip=3mm,
}
\definecolor{maroon}{rgb}{0.5,0,0}
\definecolor{darkgreen}{rgb}{0,0.5,0}
\lstdefinelanguage{XML}
{
  basicstyle=\ttfamily,
  morestring=[s]{"}{"},
  morecomment=[s]{?}{?},
  morecomment=[s]{!--}{--},
  commentstyle=\color{darkgreen},
  moredelim=[s][\color{black}]{>}{<},
  %moredelim=[s][\color{red}]{\ }{=},
  stringstyle=\color{blue},
  identifierstyle=\color{maroon}
}

\pagestyle{fancy} 
\rhead{February 2015} 
\lhead{Wave Media}
\title{SmartSlides Project Wide Standard}
\author{T. Davidson \\ WaveMedia}


\begin{document}
\maketitle

\clearpage
\tableofcontents
\clearpage

\section{Rules}
\subsection{XML Structure}
\begin{itemize}
\item The top level element is the slideshow element.
\item The slideshow element must include both documentinfo and deafualtsettings elements. These elements must occur in the order documentinfo, deafultsettings.
\item The slideshow element must include 1 or more slide elements.
\item The documentinfo element must include author, version, comment and groupid elements. These elements may occur in any order.
\item The defaultsettings element must include backgroundcolor, font, fontsize and fontcolor elements. These elements may occur in any order.
\item The slide element can include any number of text, image, audio, video and graphics elements. These elements can occur in any order.
\item The text element must include xstart and ystart attributes.
\item The text element can include sourcefile, font, fontsize, fontcolor and duration attributes.
\item The text element can be an empty element, if a source file is specified.
\item The image element cannot include any child elements.
\item The image element must include sourcefile, xstart and ystart attributes.
\item The image element can include scale, duration and starttime attributes.
\item The audio element cannot include any child elements.
\item The audio element must include a sourcefile attribute.
\item The audio element can include a starttime attribute.
\item The video element cannot include any child elements.
\item The video element must include sourcefile, xstart and ystart attributes.
\item The graphics element must include type, xstart, ystart, xend, yend, solid and graphiccolor attributes.
\item The graphics element can include a duration attribute.
\item The graphics element can include a cyclicshading element.
\item The cylclicshading element cannot include any elements.
\item The cyclicshading element must include a shadingcolor attribute.

\end{itemize}

\clearpage

\subsection{Additional Rules}
\begin{itemize}
\item XML, and therefore the project wide standard, is case sensitive. All element names, attribute names and data must be written purely in lower case characters.
\item Color is always spelt color, not colour.
\item All colors are described as 8 digit hex strings in ARGB format. (The alpha channel is the first two digits).
\item All position values are relative to screen size, measured from the top left corner of the screen, and describe the top left corner of the object. The value is a float data type, in the range 0.0 - 1.0.
\item All duration values are given in seconds. The value is a float data type.
\item Where a duration value is not given an infinite duration is implied.
\item A group ID value of 0 specifies that the file uses the standard format, with no group's extensions.
\item All font sizes are integer pt sizes.
\item The standard supports both 'arial' and 'times new roman' fonts.
\item Slides occur in the XML file in sequential order, first to last.
\item Text source files should be of type '.txt'.
\item Image source files should be of type '.png' or '.jpg'.
\item Audio source files should be of type '.wav'. 
\item Video source files should be of type '.mkv'.
\item Image scale attribute is relative to the dimensions of the source file. The value is a float data type greater than 0.
\item The graphic 'type' attribute supports the following values: oval, rectangle, line.
\end{itemize}

\clearpage
\section{Text Handler}
Five different XML text standards need to be supported, which are shown below.

The hierarchy for settings is program defaults, slideshow defaults(if the parameter exists), text element, richtext element.

The hierarchy for what text is printed is text from sourcefile, then text within XML file.
\subsection{Text included with a null string for sourcefile}
\lstset{language=XML}
\lstinputlisting{"TextXML1.xml"}

\subsection{Text included with no sourcefile attribute}
\lstinputlisting{"TextXML2.xml"}

\subsection{Text from sourcefile}
\lstinputlisting{"TextXML3.xml"}

\subsection{Text from list of richtext elements}
\lstinputlisting{"TextXML4.xml"}
	
\clearpage

\section{Graphics Handler}
Below is a list of all shapes supported by the Graphics Handler, and the parameters that can be varied on each shape. 

\subsection{Common Parameters}
Each shape requires these values specified. 
\begin{itemize}
\item xStartPos - x coordinate of the top left corner of the bounding box around the shape.
\item yStartPos - y coordinate of the top left corner of the bounding box around the shape.
\end{itemize}


\subsection{Oval}
\begin{itemize}
\item xEndPos - x coordinate of the bottom right corner of the bounding box around the oval.
\item yEndPos - y coordinate of the bottom right corner of the bounding box around the oval.
\item color - color of the oval. The accepted string format is a \#, followed by a 8 bit hex number in ARGB format.
\item solid - boolean value of if the oval is an outline or a solid shape.
\item outlineColor - outline color of the oval. The accepted string format is a \#, followed by a 8 bit hex number in ARGB format.
\item outlineThickness - thickness of the oval outline.
\item shadowType - the amount of shadow on the oval.\newline  Options: Shadow.NONE, Shadow.LIGHT, Shadow.NORMAL and Shadow.HEAVY.
\item rotation - amount of rotation about the center in degrees. 
\item shadingType - the type of shading to be applied to the oval.\newline  Options: Shading.NONE, Shading.CYCLIC, Shading.HORIZONTAL, Shading.VERTICAL.
\item shadingElement - a element that describes the color at a position in the shape. Contains a shadingColor in the regular color string format, and a offset that is a relative (0 to 1) distance through the shape that the color should appear. This element can occur multiple times per shape.
\end{itemize}


\subsection{Circle}
\begin{itemize}
\item radius - the radius of the circle.
\item color - color of the circle. The accepted string format is a \#, followed by a 8 bit hex number in ARGB format.
\item solid - boolean value of if the circle is an outline or a solid shape.
\item outlineColor - outline color of the circle. The accepted string format is a \#, followed by a 8 bit hex number in ARGB format.
\item outlineThickness - thickness of the circle outline.
\item shadowType - the amount of shadow on the circle.\newline  Options: Shadow.NONE, Shadow.LIGHT, Shadow.NORMAL and Shadow.HEAVY.
\item shadingType - the type of shading to be applied to the circle.\newline  Options: Shading.NONE, Shading.CYCLIC, Shading.HORIZONTAL, Shading.VERTICAL.
\item shadingElement - a element that describes the color at a position in the shape. Contains a shadingColor in the regular color string format, and a offset that is a relative (0 to 1) distance through the shape that the color should appear. This element can occur multiple times per shape.
\end{itemize}

\subsection{Rectangle}
\begin{itemize}
\item xEndPos - x coordinate of the bottom right corner of the bounding box around the rectangle.
\item yEndPos - y coordinate of the bottom right corner of the bounding box around the rectangle.
\item arcWidth - the vertical diameter of the arc at the four corners of the rectangle.
\item arcHeight - the horizontal diameter of the arc at the four corners of the rectangle.
\item color - color of the rectangle. The accepted string format is a \#, followed by a 8 bit hex number in ARGB format.
\item solid - boolean value of if the rectangle is an outline or a solid shape.
\item outlineColor - outline color of the rectangle. The accepted string format is a \#, followed by a 8 bit hex number in ARGB format.
\item outlineThickness - thickness of the rectangle outline.
\item shadowType - the amount of shadow on the rectangle.\newline  Options: Shadow.NONE, Shadow.LIGHT, Shadow.NORMAL and Shadow.HEAVY.
\item rotation - amount of rotation about the center in degrees. 
\item shadingType - the type of shading to be applied to the rectangle. \newline Options: Shading.NONE, Shading.CYCLIC, Shading.HORIZONTAL, Shading.VERTICAL.
\item shadingElement - a element that describes the color at a position in the shape. Contains a shadingColor in the regular color string format, and a offset that is a relative (0 to 1) distance through the shape that the color should appear. This element can occur multiple times per shape.
\end{itemize}

\subsection{Square}
\begin{itemize}
\item length - the length of each side of the square.
\item color - color of the rectangle. The accepted string format is a \#, followed by a 8 bit hex number in ARGB format.
\item solid - boolean value of if the square is an outline or a solid shape.
\item outlineColor - outline color of the square. The accepted string format is a \#, followed by a 8 bit hex number in ARGB format.
\item outlineThickness - thickness of the square outline.
\item shadowType - the amount of shadow on the square.\newline  Options: Shadow.NONE, Shadow.LIGHT, Shadow.NORMAL and Shadow.HEAVY.
\item rotation - amount of rotation about the center in degrees. 
\item shadingType - the type of shading to be applied to the square. \newline  Options: Shading.NONE, Shading.CYCLIC, Shading.HORIZONTAL, Shading.VERTICAL.
\item shadingElement - a element that describes the color at a position in the shape. Contains a shadingColor in the regular color string format, and a offset that is a relative (0 to 1) distance through the shape that the color should appear. This element can occur multiple times per shape.
\end{itemize}

\subsection{Line}
\begin{itemize}
\item xStartPos - x coordinate of the start of the line.
\item yStartPos - y coordinate of the start of the line.
\item xEndPos - x coordinate of the end of the line.
\item yEndPos - y coordinate of the end of the line.
\item color - color of the line. The accepted string format is a \#, followed by a 8 bit hex number in ARGB format.
\item thickness - thickness of the line.
\item shadingType - the type of shading to be applied to the line.\newline  Options: Shading.NONE, Shading.CYCLIC, Shading.HORIZONTAL, Shading.VERTICAL.
\item shadingElement - a element that describes the color at a position in the shape. Contains a shadingColor in the regular color string format, and a offset that is a relative (0 to 1) distance through the shape that the color should appear. This element can occur multiple times per shape.
\end{itemize}

\subsection{Arrow}
\begin{itemize}
\item xStartPos - x coordinate of the start of the arrow.
\item yStartPos - y coordinate of the start of the arrow.
\item xEndPos - x coordinate of the end of the arrow.
\item yEndPos - y coordinate of the end of the arrow.
\item color - color of the arrow. The accepted string format is a \#, followed by a 8 bit hex number in ARGB format.
\item shadingType - the type of shading to be applied to the arrow.\newline  Options: Shading.NONE, Shading.CYCLIC, Shading.HORIZONTAL, Shading.VERTICAL.
\item shadingElement - a element that describes the color at a position in the shape. Contains a shadingColor in the regular color string format, and a offset that is a relative (0 to 1) distance through the shape that the color should appear. This element can occur multiple times per shape.
\end{itemize}


\subsection{Equilateral Triangle}
\begin{itemize}
\item length - the side length of the triangle
\item color - color of the triangle. The accepted string format is a \#, followed by a 8 bit hex number in ARGB format.
\item solid - boolean value of if the triangle is an outline or a solid shape.
\item outlineColor - outline color of the triangle. The accepted string format is a \#, followed by a 8 bit hex number in ARGB format.
\item outlineThickness - thickness of the triangle outline.
\item shadowType - the amount of shadow on the triangle.\newline  Options: Shadow.NONE, Shadow.LIGHT, Shadow.NORMAL and Shadow.HEAVY.
\item rotation - amount of rotation about the center in degrees. 
\item shadingType - the type of shading to be applied to the equilateral triangle.\newline  Options: Shading.NONE, Shading.CYCLIC, Shading.HORIZONTAL, Shading.VERTICAL.
\item shadingElement - a element that describes the color at a position in the shape. Contains a shadingColor in the regular color string format, and a offset that is a relative (0 to 1) distance through the shape that the color should appear. This element can occur multiple times per shape.
\end{itemize}


\subsection{Triangle}
\begin{itemize}
\item triangleCoordinates - all 6 coordinates of the 3 corners of the triangle.
\item color - color of the triangle. The accepted string format is a \#, followed by a 8 bit hex number in ARGB format.
\item solid - boolean value of if the triangle is an outline or a solid shape.
\item outlineColor - outline color of the triangle. The accepted string format is a \#, followed by a 8 bit hex number in ARGB format.
\item outlineThickness - thickness of the triangle outline.
\item shadowType - the amount of shadow on the triangle.\newline  Options: Shadow.NONE, Shadow.LIGHT, Shadow.NORMAL and Shadow.HEAVY.
\item rotation - amount of rotation about the center in degrees. 
\item shadingType - the type of shading to be applied to the triangle.\newline  Options: Shading.NONE, Shading.CYCLIC, Shading.HORIZONTAL, Shading.VERTICAL.
\item shadingElement - a element that describes the color at a position in the shape. Contains a shadingColor in the regular color string format, and a offset that is a relative (0 to 1) distance through the shape that the color should appear. This element can occur multiple times per shape.
\end{itemize}


\subsection{N Sided Regular Polygon}
\begin{itemize}
\item width - the width of the polygon.
\item height - the height of the polygon.
\item numberOfSides - the number of sides of the polygon.
\item color - color of the polygon. The accepted string format is a \#, followed by a 8 bit hex number in ARGB format.
\item solid - boolean value of if the polygon is an outline or a solid shape.
\item outlineColor - outline color of the polygon. The accepted string format is a \#, followed by a 8 bit hex number in ARGB format.
\item outlineThickness - thickness of the polygon outline.
\item shadowType - the amount of shadow on the polygon.\newline  Options: Shadow.NONE, Shadow.LIGHT, Shadow.NORMAL and Shadow.HEAVY.
\item rotation - amount of rotation about the center in degrees. 
\item shadingType - the type of shading to be applied to the polygon.\newline  Options: Shading.NONE, Shading.CYCLIC, Shading.HORIZONTAL, Shading.VERTICAL.
\item shadingElement - a element that describes the color at a position in the shape. Contains a shadingColor in the regular color string format, and a offset that is a relative (0 to 1) distance through the shape that the color should appear. This element can occur multiple times per shape.
\end{itemize}


\subsection{Polygon}
\begin{itemize}
\item polygonCoordinate - coordinates of one of the corners of the polygon.
\item xStartPos - x coordinate of the top left corner of the bounding box around the polygon. All the x coordinates are relative to this position.
\item yStartPos - y coordinate of the top left corner of the bounding box around the polygon. All the y coordinates are relative to this position.
\item color - color of the polygon. The accepted string format is a \#, followed by a 8 bit hex number in ARGB format.
\item solid - boolean value of if the polygon is an outline or a solid shape.
\item outlineColor - outline color of the polygon. The accepted string format is a \#, followed by a 8 bit hex number in ARGB format.
\item outlineThickness - thickness of the polygon outline.
\item shadowType - the amount of shadow on the polygon.\newline  Options: Shadow.NONE, Shadow.LIGHT, Shadow.NORMAL and Shadow.HEAVY.
\item rotation - amount of rotation about the center in degrees. 
\item shadingType - the type of shading to be applied to the polygon.\newline  Options: Shading.NONE, Shading.CYCLIC, Shading.HORIZONTAL, Shading.VERTICAL.
\item shadingElement - a element that describes the color at a position in the shape. Contains a shadingColor in the regular color string format, and a offset that is a relative (0 to 1) distance through the shape that the color should appear. This element can occur multiple times per shape.
\end{itemize}


\subsection{Star}
\begin{itemize}
\item numberOfPoints - the number of points of the star. 
\item color - color of the star. The accepted string format is a \#, followed by a 8 bit hex number in ARGB format.
\item solid - boolean value of if the star is an outline or a solid shape.
\item outlineColor - outline color of the star. The accepted string format is a \#, followed by a 8 bit hex number in ARGB format.
\item outlineThickness - thickness of the star outline.
\item shadowType - the amount of shadow on the star.\newline  Options: Shadow.NONE, Shadow.LIGHT, Shadow.NORMAL and Shadow.HEAVY.
\item rotation - amount of rotation about the center in degrees. 
\item shadingType - the type of shading to be applied to the star.\newline  Options: Shading.NONE, Shading.CYCLIC, Shading.HORIZONTAL, Shading.VERTICAL.
\item shadingElement - a element that describes the color at a position in the shape. Contains a shadingColor in the regular color string format, and a offset that is a relative (0 to 1) distance through the shape that the color should appear. This element can occur multiple times per shape.
\end{itemize}


\subsection{Chord}
\begin{itemize}
\item width - the overall width (horizontal radius) of the full ellipse of which the arc in the chord is a partial section. 
\item height - the overall height (vertical radius) of the full ellipse of which the arc in the chord is a partial section.
\item arcAngle - the starting angle of the arc in the chord in degrees.
\item length - the angular extent of the arc in the chord in degrees.
\item color - color of the chord. The accepted string format is a \#, followed by a 8 bit hex number in ARGB format.
\item solid - boolean value of if the chord is an outline or a solid shape.
\item outlineColor - outline color of the chord. The accepted string format is a \#, followed by a 8 bit hex number in ARGB format.
\item outlineThickness - thickness of the chord outline.
\item shadowType - the amount of shadow on the chord.\newline  Options: Shadow.NONE, Shadow.LIGHT, Shadow.NORMAL and Shadow.HEAVY.
\item rotation - amount of rotation about the center in degrees. 
\item shadingType - the type of shading to be applied to the chord.\newline  Options: Shading.NONE, Shading.CYCLIC, Shading.HORIZONTAL, Shading.VERTICAL.
\item shadingElement - a element that describes the color at a position in the shape. Contains a shadingColor in the regular color string format, and a offset that is a relative (0 to 1) distance through the shape that the color should appear. This element can occur multiple times per shape.
\end{itemize}


\subsection{Arc}
\begin{itemize}
\item width - the overall width (horizontal radius) of the full ellipse of which this arc is a partial section. 
\item height - the overall height (vertical radius) of the full ellipse of which this arc is a partial section.
\item arcAngle - the starting angle of the arc in degrees.
\item length - the angular extent of the arc in degrees.
\item color - color of the arc. The accepted string format is a \#, followed by a 8 bit hex number in ARGB format.
\item solid - boolean value of if the arc is an outline or a solid shape.
\item outlineColor - outline color of the arc. The accepted string format is a \#, followed by a 8 bit hex number in ARGB format.
\item outlineThickness - thickness of the arc outline.
\item shadowType - the amount of shadow on the arc.\newline  Options: Shadow.NONE, Shadow.LIGHT, Shadow.NORMAL and Shadow.HEAVY.
\item rotation - amount of rotation about the center in degrees. 
\item shadingType - the type of shading to be applied to the rectangle. Options: Shading.NONE, Shading.CYCLIC, Shading.HORIZONTAL, Shading.VERTICAL.
\item shadingElement - a element that describes the color at a position in the shape. Contains a shadingColor in the regular color string format, and a offset that is a relative (0 to 1) distance through the shape that the color should appear. This element can occur multiple times per shape.
\end{itemize}


\section{Image Handler}
\clearpage


\newgeometry{left=1cm,right=1cm}
\section{Example XML file}
\small
\lstset{language=XML}
\lstinputlisting{"Example PWS XML.xml"}

\clearpage

\section{Schema}
\lstset{language=XML}
\lstinputlisting{"PWS Schema.xsd"}

\end{document}
\documentclass[oneside]{article}
\usepackage[utf8]{inputenc}
\usepackage{parskip}
\usepackage[UKenglish]{babel}
\usepackage[margin=1in]{geometry}
\usepackage{fancyhdr}
\usepackage{multirow}
\usepackage[colorlinks=false]{hyperref}
\usepackage[super,square]{natbib}
\usepackage{float}
\usepackage[toc,page]{appendix}
\usepackage[table]{xcolor}


\usepackage{color}
\usepackage{listings}
\usepackage{graphicx}
\usepackage{caption}
\usepackage{wrapfig}
\usepackage{lscape}
\usepackage{rotating}
\usepackage{epstopdf}
\definecolor{mygreen}{RGB}{0,127,0}
\definecolor{mygray}{RGB}{100,100,100}
\definecolor{mymauve}{RGB}{100,32,255}
\definecolor{lgray}{RGB}{230,230,230}
\definecolor{pastelgreen}{RGB}{119, 190, 119}
\definecolor{pastelred}{RGB}{255, 105, 97}
\definecolor{pastelorange}{RGB}{255, 179, 71}
\lstset{ %
  frame=tb,
  backgroundcolor=\color{white},   % choose the background color; you must add \usepackage{color} or \usepackage{xcolor}
  basicstyle=\footnotesize\ttfamily,        % the size of the fonts that are used for the code
  breakatwhitespace=false,         % sets if automatic breaks should only happen at whitespace
  breaklines=true,                 % sets automatic line breaking
  captionpos=t,                    % sets the caption-position to bottom
  commentstyle=\color{mygreen},    % comment style
  deletekeywords={...},            % if you want to delete keywords from the given language
  escapeinside={\%*}{*)},          % if you want to add LaTeX within your code
  extendedchars=true,              % lets you use non-ASCII characters; for 8-bits encodings only, does not work with UTF-8
%  frame=single,                    % adds a frame around the code
  keepspaces=true,                 % keeps spaces in text, useful for keeping indentation of code (possibly needs columns=flexible)
  keywordstyle=\color{blue},       % keyword style
  language=,                 % the language of the code
  morekeywords={*,...},            % if you want to add more keywords to the set
  numbers=left,                    % where to put the line-numbers; possible values are (none, left, right)
  numbersep=5pt,                   % how far the line-numbers are from the code
  numberstyle=\tiny\color{mygray}, % the style that is used for the line-numbers
  rulecolor=\color{black},         % if not set, the frame-color may be changed on line-breaks within not-black text (e.g. comments (green here))
  showspaces=false,                % show spaces everywhere adding particular underscores; it overrides 'showstringspaces'
  showstringspaces=false,          % underline spaces within strings only
  showtabs=false,                  % show tabs within strings adding particular underscores
  stepnumber=1,                    % the step between two line-numbers. If it's 1, each line will be numbered
  stringstyle=\color{mymauve},     % string literal style
  tabsize=4,                       % sets default tabsize to 2 spaces
  aboveskip=3mm,
  belowskip=3mm,
}
\definecolor{maroon}{rgb}{0.5,0,0}
\definecolor{darkgreen}{rgb}{0,0.5,0}
\lstdefinelanguage{XML}
{
  basicstyle=\ttfamily,
  morestring=[s]{"}{"},
  morecomment=[s]{?}{?},
  morecomment=[s]{!--}{--},
  commentstyle=\color{darkgreen},
  moredelim=[s][\color{black}]{>}{<},
  %moredelim=[s][\color{red}]{\ }{=},
  stringstyle=\color{blue},
  identifierstyle=\color{maroon}
}




\pagestyle{fancy} 
\rhead{February 2015} 
\lhead{Wave Media}
\title{Project Wide Standard}
\author{T. Davidson}

\begin{document}
\maketitle

\clearpage

\section{Rules}
\subsection{XML Structure}
\begin{itemize}
\item The top level element is the slideshow element.
\item The slideshow element must include both documentinfo and deafualtsettings elements. These elements must occur in the order documentinfo, deafultsettings.
\item The slideshow element must include 1 or more slide elements.
\item The documentinfo element must include author, version, comment and groupid elements. These elements may occur in any order.
\item The defaultsettings element must include backgroundcolor, font, fontsize and fontcolor elements. These elements may occur in any order.
\item The slide element can include any number of text, image, audio, video and graphics elements. These elements can occur in any order.
\item The text element must include xstart and ystart attributes.
\item The text element can include sourcefile, font, fontsize, fontcolor and duration attributes.
\item The text element can be an empty element, if a source file is specified.
\item The image element cannot include any child elements.
\item The image element must include sourcefile, xstart and ystart attributes.
\item The image element can include scale, duration and starttime attributes.
\item The audio element cannot include any child elements.
\item The audio element must include a sourcefile attribute.
\item The audio element can include a starttime attribute.
\item The video element cannot include any child elements.
\item The video element must include sourcefile, xstart and ystart attributes.
\item The graphics element must include type, xstart, ystart, xend, yend, solid and graphiccolor attributes.
\item The graphics element can include a duration attribute.
\item The graphics element can include a cyclicshading element.
\item The cylclicshading element cannot include any elements.
\item The cyclicshading element must include a shadingcolor attribute.

\end{itemize}

\clearpage

\subsection{Additional Rules}
\begin{itemize}
\item XML, and therefore the project wide standard, is case sensitive. All element names, attribute names and data must be written purely in lower case characters.
\item Color is always spelt color, not colour.
\item All colors are described as 8 digit hex strings in ARGB format. (The alpha channel is the first two digits).
\item All position values are relative to screen size, measured from the top left corner of the screen, and describe the top left corner of the object. The value is a float data type, in the range 0.0 - 1.0.
\item All duration values are given in seconds. The value is a float data type.
\item Where a duration value is not given an infinite duration is implied.
\item A group ID value of 0 specifies that the file uses the standard format, with no group's extensions.
\item All font sizes are integer pt sizes.
\item The standard supports both 'arial' and 'times new roman' fonts.
\item Slides occur in the XML file in sequential order, first to last.
\item Text source files should be of type '.txt'.
\item Image source files should be of type '.png' or '.jpg'.
\item Audio source files should be of type '.wav'. 
\item Video source files should be of type '.mkv'.
\item Image scale attribute is relative to the dimensions of the source file. The value is a float data type greater than 0.
\item The graphic 'type' attribute supports the following values: oval, rectangle, line.
\end{itemize}

\clearpage

\newgeometry{left=1cm,right=1cm}
\section{Example XML file}
\small
\lstset{language=XML}
\lstinputlisting{"Example PWS XML.xml"}

\clearpage

\section{Schema}
\lstset{language=XML}
\lstinputlisting{"PWS Schema.xsd"}

\end{document}